\section{Introduction}\label{sec:Introduction}

Over the last few years, manufacturing companies have experienced an ever-increasing demand for more complex products with increasing product variants. As a result, the number of tasks assembly operators must master is growing accordingly. The introduction of Industry 4.0 concepts and technologies also requires operators to acquire new skills (e.g., robot programming, data analysis) linked to their changed roles and responsibilities within the production environment. The impact of this change cannot be underestimated and is, therefore, one of the main focus points for innovation in the recently defined Industry 5.0 concept. 

Currently, assembly operator support services are often limited to procedural (digital) work instructions. In case of issues, operators must rely on their knowledge and experience or require help from external experts. Digitization offers an enormous potential to provide better insights and support to the operator by gathering information and knowledge that is available in various IT information systems. However, today, this potential largely remains untapped. 

In this project, we will leverage this knowledge potential to develop a digital assistant (DA) that interacts with the operator and serves as a single-point-of-contact between the operator and various information sources. Based on accurate recognition of the current state of the assembly, the DA will proactively suggest what the operator should do next or what the operator should do differently. When issues arise, the operator will be able to seamlessly interact with the DA through various devices and modalities (e.g., \textit{''Why doesn't this part fit?''}), and the DA relies on its understanding of the situation to provide the operator with answers. 

Besides performance gains (e.g., regarding quality and efficiency), such a DA also provides reassurance to operators, as they know that they will receive the required support whenever they need it. Furthermore, new operators will be able to work independently more quickly, and the decreased need for external support will increase efficiency of the supportive staff.
