%% The first command in your LaTeX source must be the \documentclass command.
%%
%% Options:
%% twocolumn : Two column layout.
%% hf: enable header and footer.
\documentclass[
% twocolumn,
% hf,
]{ceurart}

%%
%% One can fix some overfulls
\sloppy

%%
%% Minted listings support 
%% Need pygment <http://pygments.org/> <http://pypi.python.org/pypi/Pygments>
\usepackage{listings}
%% auto break lines
\lstset{breaklines=true}
\usepackage{graphicx}
\graphicspath{{figs/}}
%%
%% end of the preamble, start of the body of the document source.
\begin{document}

%%
%% Rights management information.
%% CC-BY is default license.
\copyrightyear{2022}
\copyrightclause{Copyright for this paper by its authors.
  Use permitted under Creative Commons License Attribution 4.0
  International (CC BY 4.0).}

%%
%% This command is for the conference information
\conference{Caise'24: 36th International Conference on Advanced Information Systems Engineering, June 03--07, 2024, Limassol, Cyprus}

%%
%% The "title" command
\title{Operator Assist: A digital assistant for assembly operators}

%\tnotemark[1]
%\tnotetext[1]{You can use this document as the template for preparing your publication. We recommend using the latest version of the ceurart style.}

%%
%% The "author" command and its associated commands are used to define
%% the authors and their affiliations.
%\author[1,2]{Dmitry S. Kulyabov}[%
%orcid=0000-0002-0877-7063,
%email=kulyabov-ds@rudn.ru,
%url=https://yamadharma.github.io/,
%]
%\cormark[1]
%\fnmark[1]
%\address[1]{Peoples' Friendship University of Russia (RUDN University),
%  6 Miklukho-Maklaya St, Moscow, 117198, Russian Federation}
%\address[2]{Joint Institute for Nuclear Research,
%  6 Joliot-Curie, Dubna, Moscow region, 141980, Russian Federation}
%
%\author[3]{Ilaria Tiddi}[%
%orcid=0000-0001-7116-9338,
%email=i.tiddi@vu.nl,
%url=https://kmitd.github.io/ilaria/,
%]
%\fnmark[1]
%\address[3]{Vrije Universiteit Amsterdam, De Boelelaan 1105, 1081 HV Amsterdam, The Netherlands}
%
%\author[4]{Manfred Jeusfeld}[%
%orcid=0000-0002-9421-8566,
%email=Manfred.Jeusfeld@acm.org,
%url=http://conceptbase.sourceforge.net/mjf/,
%]
%\fnmark[1]
%\address[4]{University of Skövde, Högskolevägen 1, 541 28 Skövde, Sweden}

%% Footnotes
\cortext[1]{Corresponding author.}
\fntext[1]{These authors contributed equally.}

%%
%% The abstract is a short summary of the work to be presented in the
%% article.
\begin{abstract}
This research project addresses the evolving challenges faced by manufacturing companies in response to a rising demand for intricate products and an expanding array of product variants. The implementation of Industry 4.0 concepts has necessitated assembly operators to acquire additional skills related to their altered roles, leading to a paradigm shift outlined in Industry 5.0. Current support services for assembly operators primarily offer procedural digital instructions, leaving operators to rely on personal knowledge or seek external assistance when issues arise. The untapped potential of digitization in consolidating information from diverse IT systems can significantly enhance operator support.
This project aims to harness this knowledge potential by developing a digital assistant (DA) as a central interface between operators and information sources. The DA, guided by a precise understanding of the assembly's current state, will proactively recommend the next steps for the operator. In instances of challenges, seamless interaction with the DA will provide prompt answers based on its situational comprehension. Beyond enhancing performance metrics such as quality and efficiency, the DA offers operators a sense of reassurance by ensuring timely support. Additionally, it facilitates quicker independence for new operators and boosts supportive staff efficiency by reducing the reliance on external assistance.
\end{abstract}

%%
%% Keywords. The author(s) should pick words that accurately describe
%% the work being presented. Separate the keywords with commas.
\begin{keywords}
  OperatorAssist\_SBO\sep 
  Operator Assist \sep Digital Assistant (DA) \sep
  Digital work instruction (DWI) \sep
  Question and answers \sep
  Vision system \sep
  Knowledge graph
\end{keywords}

%%
%% This command processes the author and affiliation and title
%% information and builds the first part of the formatted document.
\maketitle

\section{Introduction}\label{sec:Introduction}

Over the last few years, manufacturing companies have experienced an ever-increasing demand for more complex products with increasing product variants. As a result, the number of tasks assembly operators must master is growing accordingly. The introduction of Industry 4.0 concepts and technologies also requires operators to acquire new skills (e.g., robot programming, data analysis) linked to their changed roles and responsibilities within the production environment. The impact of this change cannot be underestimated and is, therefore, one of the main focus points for innovation in the recently defined Industry 5.0 concept. 

Currently, assembly operator support services are often limited to procedural (digital) work instructions. In case of issues, operators must rely on their knowledge and experience or require help from external experts. Digitization offers an enormous potential to provide better insights and support to the operator by gathering information and knowledge that is available in various IT information systems. However, today, this potential largely remains untapped. 

In this project, we will leverage this knowledge potential to develop a digital assistant (DA) that interacts with the operator and serves as a single-point-of-contact between the operator and various information sources. Based on accurate recognition of the current state of the assembly, the DA will proactively suggest what the operator should do next or what the operator should do differently. When issues arise, the operator will be able to seamlessly interact with the DA through various devices and modalities (e.g., \textit{''Why doesn't this part fit?''}), and the DA relies on its understanding of the situation to provide the operator with answers. 

Besides performance gains (e.g., regarding quality and efficiency), such a DA also provides reassurance to operators, as they know that they will receive the required support whenever they need it. Furthermore, new operators will be able to work independently more quickly, and the decreased need for external support will increase efficiency of the supportive staff.

\section{Related works}\label{sec:Literature}

The application of DAs within manufacturing environments has started being reported in recent literature [6]� [8]. In 2019, Gartner predicted that by the end of this year over 25\% of so-called digital workers will make daily use of a virtual assistant of some sort [9]. Despite encouraging first steps, with positive POC results [10], a number of characteristics specific to assembly environments lead to unresolved challenges that inhibit a wide rollout of DAs in these environments. Take a simple example where we consider a scenario in which the DA wants to support the operator in his next task. We want the DA to answer to the question “what is the next task” as well as to proactively answer to the question of the operator “why do I have an issue with executing this task”. Thinking through this example, the following challenges surface.

In order for the DA to determine the next task, the DA should understand what is going on. Awareness on the current state of the assembly process to deduce what potential follow-up actions could be. Because of the size and dynamics of the assembly environment and the freedoms the operators often have, capturing this assembly state is challenging. The DA will also require two-way interaction with the operator to gain further contextual information and to suggest support actions. Due to the heterogeneity of working environments, it is impossible to define one generic User Interface (UI) to provide that seamless interaction. E.g. in noisy environments, a touch-based interface is preferred over interaction through speech, while in other settings the higher information throughput of speech might be preferred. Also, tasks that require the use of both hands and touch-based interaction through tablets are impossible, while this might be ideal in circumstances in which the operator needs to be mobile.

Lastly, the DA requires intelligence to provide answers. This encompasses two aspects: first, a deep and clear understanding and knowledge of the assembly process and environment. This knowledge can only be obtained through the integration of and information sharing between various, currently unlinked and unstructured information sources. Second, the DA requires the intelligence to reason on this knowledge and link it to the perceived context to formulate answers to understand the ‘why’ question and provide answers. DA solutions in other domains are based on data-driven methods that rely on vast (crowd-sourced) training datasets to provide this kind of reasoning intelligence and knowledge to the system. In assembly environments, especially in the case of high variability and low volume, these datasets are not always available because of low occurrence of some tasks.
\section{Methodology}\label{sec:Methodology}

...

\subsection{Digital Assistant}\label{sec:DA}

Speech, which is the prevalent way to interact with a DA, is characterized by high efficiency, naturalness, low
cognitive load and hands-free capabilities, but is hard when it is noisy or multiple persons are working in close
proximity [23]. Speech, however, can be complemented by graphical UIs to query a DA, for instance through a
chatbot and touch/gestural interaction [7], [24], [25]. Leveraging multiple modalities and devices in combination with
a conversational UI allows seamless interaction [26], as well as taking advantage of fitting technologies in different
situations [27]. Therefore, we will support multiple devices and modalities to interact with the DA, using web-based
multi-device formats. Interaction between the operator and DA will always happen through a UI on a device, which
needs to be created for and tailored to that specific device for an optimal experience. Automatic UI creation and
adaptation mechanisms, however, are error prone and complex, and lack control, transparency and predictability
[28]. On the work floor, operators require stable interfaces that are predictable. Therefore, we focus on an adaptable
UI [29] and design templates that take into account best practices [30]to ensure a consistent and usable experience.
In contrast to existing work, we will use a holistic approach that considers (i) context, devices, modalities, and UI
together, (ii) from both a technical and usability perspective, (iii) in a manufacturing setting: we will provide a toolbox
for multimodal UIs that can be rendered on multiple devices and support seamless two-way interaction, which can
be ‘optimized’ for the current manufacturing environment, task and operator (see figure 1 for a practical example).
We will build on our experiences with, among others, asking and answering questions about computing applications
[31].
A major benefit of the DA is the on-demand, reactive nature of support, as nonstop support can negatively influence
task completion time and perceived cognitive load [32]. The downside is that info might be overlooked by the
operator. Therefore, the DA will offer proactive assistance when needed (e.g., in case of errors in task execution)
[33], [34]. We will investigate decision strategies that use implicit inputs such as cognitive load [35] and state
estimation. Proactivity, however, will always be imperfect, akin to inevitable flaws in activity and context detection:
we will implement a two-way feedback loop, so the operator can complement or correct info (e.g., in case of low
confidence level of an activity recognition or additional information offered because of a lack of experience, the
operator is made aware and can intervene). To process both explicit and implicit inputs, we will build on our
experiences with prototyping multimodal interactions [36], as it is not our goal to advance the state of the art with
regard to low-level input processing. We will also implement strategies to attune the output to the context of use
(e.g., the operator's experience, the task's criticality, the probability of errors). To this end, we will render information
with various levels of detail [37], [38], adapted to a manufacturing setting and augmented with a mixed-initiative
approach to keep the operator in control. An inexperienced operator, for instance, will be offered more detailed
information (e.g., instructional video or diagram instead of text) and more frequent proactive suggestions, but the
operator can always intervene to avoid unwanted or unneeded behavior.

\subsection{Reactive and proactive support} \label{sec:ReactiveProactive}



\section{Current Status and Results of OperatorAssist}\label{sec:Results}

The DA is the result of interaction between four individual functional modules: the action recognition and state estimator providing contextual information, the assembly knowledge store accumulating structured assembly knowledge and contextual information, the reasoning engine mapping content to the perceived context, and the UI toolbox facilitating seamless interaction between the DA and operator.

\subsection{User Interface Toolbox}\label{sec:UI}
The key goal of the UI module is to offer a user friendly, efficient and uniform point of access to the DA, which can be optimized for the current manufacturing environment, task and operator. Therefor, we provide a toolbox for adaptable multimodal UIs that can be rendered on multiple devices and support seamless two-way interaction.

The focal element in the architectural framework behind our UI toolbox (see Figure~\ref{fig:ui-architecture}) is the back-end C\# WPF application that serves as an encapsulating entity for a SignalR hub. It functions as the central authority and communication hub, governing all communication processes initiated by individual clients. There are no limitations for the type of devices that can act as a client, apart from being able to communicate through SignalR~\cite{SignalR01,Sharma_2023}. We implemented a toolbox of reusable components on the Microsoft MAUI platform~\cite{Maui01,Maui02}, with support for an extensive set of devices (e.g., desktop systems, tablets, smartphones, smartwatches), each with their own modalities, capabilities and strengths. The modular basis of the toolbox allows easy integration of other platforms and devices, including devices with limited modalities (e.g., a microcontroller directing an industrial signal tower).

While all clients are inherently equivalent, some can be defined to support more actions than others. One such client in our framework acts as a \emph{Wizard of Oz} and allows a supervising user to change the state, context and content of the system. The primary use case for this Wizard is to be able to conduct user tests with features, that might not yet have been fully implemented, but that are emulated or manipulated by the Wizard (thus facilitating rapid prototyping).
Another example of such a client is one that acts as a \emph{Bridge to the Operator Service Bus}. Communication to the other components of the project can be done through any communication channel (e.g. MQTT, REST, ...) and the bridge handles any translations from the external data to their internal representations (if deemed useful) before triggering actions within the system.

\begin{figure}
    \centering
    \includegraphics[width=1\linewidth]{figs/UI-architecture.png}
    \caption{Architectural framework behind our UI toolbox.}
    \label{fig:ui-architecture}
\end{figure}

Due to the heterogeneity of working environments, assembly tasks and operators, we focus on a toolbox that supports adaptable UIs to ensure a consistent and usable experience in a large variety of circumstances, as discussed in Section~\ref{sec:DA}. As the context of use is a fluid concept, our toolbox aims for a maximum degree of flexibility. For instance, the operator or manufacturing company can decide on which devices will be used, depending on the given context of use and availability of devices. Regarding modalities, we aim for redundancy and flexibility for an operator to interact with the DA, as some modalities might be impaired by contextual limitations (e.g., when voice input fails because of too much ambient noise, a gesture can be used instead).

Our toolbox allows for an extensive configuration of devices and modalities (e.g., which modalities should be enabled and which output should be rendered on a particular device). Furthermore, should a device or modality suddenly become unavailable, the toolbox facilitates graceful degradation and quick reconfiguration. We implemented this flexibility by creating an \emph{Interaction Event Orchestrator (IEO)} that takes the known contextual parameters into consideration to determine the best device to respond to a given input interaction. The IEO requires a device to register its modalities and capabilities, ensuring it has the necessary information available.

We use the data provided by the clients to map all the possible interactions (and their corresponding modalities) at a certain moment in time and provide it to the operator. The available interactions will vary, depending on the state and content of the application (e.g., a thumbs up gesture will signal a positive feedback to provided assistance, but might not have any effect when the operator is reading a manual). 
Additionally, after an interaction is handled, the IEO also provides information about what just happened (and by which device) so the operator can learn about the system and diagnose any unexpected results. In future work, this information could then be used to adapt and fine-tune the system's response for future events.

In addition to the reactive nature of the system, the platform also supports providing proactive assistance, which entails different notification types and urgency levels. 
The operator could, for example, receive an informative message that appears in a queue to be read at a time of the operator's discretion; the system will not try to get the attention of the operator in such an event. 
Alternatively a critical warning might need to be delivered to the operator because a potentially dangerous action was detected. In the latter case, an appropriate set of the available devices and modalities should be used to get the attention of the operator as soon as possible.
Another category of notifications are those that are quite important but not very urgent, and are therefore not warranted to interrupt the operator in his current task. This information should rather be communicated to the operator at a more \textit{opportune time}, which can be approximated by processing information on the user’s current environment \cite{lindlbauer2019ContextAwareOnlineAdaptation}, the state estimation of the task and proxemics \cite{Marquardt_2015} \cite{Williamson_2022}. 
To avoid cognitive overload, it might be a good idea to group the available notifications so the interruption, that could not be prevented, is at least used optimally.
Furthermore the perceived progress of the state estimation might give a good lead to find a natural breakpoint in the execution of the task at hand.
From the state estimator we also have access to the coordinates of the operator and the devices in the environment, which enables us to determine which devices are still within the active range (i.e., in view and/or in interaction range) of the operator and, in contrast, which can be considered as a dormant device. These dormant devices can then be handled differently (or even ignored) when trying to find the best device to signal a notification to the operator.

\iffalse
\textit{A major benefit of the DA is the on-demand, reactive nature of support, as nonstop support can negatively influence task completion time and perceived cognitive load [32]. The downside is that info might be overlooked by the operator. Therefore, the DA will offer proactive assistance when needed (e.g., in case of errors in task execution) [33], [34]. We will investigate decision strategies that use implicit inputs such as cognitive load [35], state estimation and proxemics\textcolor{red}{refs}. 
Proactivity, however, will always be imperfect, akin to inevitable flaws in activity and context detection: we will implement a two-way feedback loop, so the operator can complement or correct info (e.g., in case of low confidence level of an activity recognition or additional information offered because of a lack of experience, the operator is made aware and can intervene). }

\textit{We will also implement strategies to attune the output to the context of use (e.g., the operator's experience, the task's criticality, the probability of errors). To this end, we will render information with various levels of detail [37], [38], adapted to a manufacturing setting and augmented with a mixed-initiative approach to keep the operator in control. An inexperienced operator, for instance, will be offered more detailed information (e.g., instructional video or diagram instead of text) and more frequent proactive suggestions, but the operator can always intervene to avoid unwanted or unneeded behavior.}


Leveraging multiple modalities and devices in combination with a conversational UI allows seamless interaction [26], as well as taking advantage of fitting technologies in different situations [27]. Therefore, we will support multiple devices and modalities to interact with the DA, using web-based multi-device formats. Interaction between the operator and DA will always happen through a UI on a device, which needs to be created for and tailored to that specific device for an optimal experience. Automatic UI creation and adaptation mechanisms, however, are error prone and complex, and lack control, transparency and predictability [28]. On the work floor, operators require stable interfaces that are predictable. Therefore, we focus on an adaptable
UI [29] and design templates that take into account best practices [30]to ensure a consistent and usable experience. 
In contrast to existing work, we will use a holistic approach that considers (i) context, devices, modalities, and UI together, (ii) from both a technical and usability perspective, (iii) in a manufacturing setting: we will provide a toolbox for multimodal UIs that can be rendered on multiple devices and support seamless two-way interaction, which can be ‘optimized’ for the current manufacturing environment, task and operator (see figure 1 for a practical example).



A major benefit of the DA is the on-demand, reactive nature of support, as nonstop support can negatively influence task completion time and perceived cognitive load [32]. The downside is that info might be overlooked by the operator. Therefore, the DA will offer proactive assistance when needed (e.g., in case of errors in task execution) [33], [34]. We will investigate decision strategies that use implicit inputs such as cognitive load [35] and state estimation. Proactivity, however, will always be imperfect, akin to inevitable flaws in activity and context detection: we will implement a two-way feedback loop, so the operator can complement or correct info (e.g., in case of low
confidence level of an activity recognition or additional information offered because of a lack of experience, the operator is made aware and can intervene). 

To process both explicit and implicit inputs, we will build on our experiences with prototyping multimodal interactions [36], as it is not our goal to advance the state of the art with regard to low-level input processing. We will also implement strategies to attune the output to the context of use (e.g., the operator's experience, the task's criticality, the probability of errors). To this end, we will render information with various levels of detail [37], [38], adapted to a manufacturing setting and augmented with a mixed-initiative approach to keep the operator in control. An inexperienced operator, for instance, will be offered more detailed information (e.g., instructional video or diagram instead of text) and more frequent proactive suggestions, but the operator can always intervene to avoid unwanted or unneeded behavior.



Topics to handle:
\begin{itemize}
    \item Multi device
    \begin{itemize}
        \item Flexibility (use it or don't)
    \end{itemize}
    \begin{itemize}
        \item Graceful degradation (a device was removed from the scene)
    \end{itemize}
    \item multimodal
    \begin{itemize}
        \item redundancy (e.g. if voice doesn't work, use a gesture/...)
        \item provide insight into what's possible at a certain time (with which modalities)
        \item provide information about what just happened
    \end{itemize}
    \item give insight into current assembly state
    \begin{itemize}
        \item 2 way feedback loop
    \end{itemize}
    \begin{itemize}
        \item proxemics to take position of the operator into account?
    \end{itemize}
    \item Reactive <> proactive
    \begin{itemize}
        \item Proactivity:
        \begin{itemize}
            \item interruptability - opportune time to interrupt the user
        \end{itemize}
    \end{itemize}
\end{itemize}
\fi
\subsection{Reasoning Engine}\label{sec:RE}

...Content
\input{04_Results_ASE}
\subsection{Assembly Knowledge Store}\label{sec:AKS}

...

\input{05_Conclusions}

%%
%% The acknowledgments section is defined using the "acknowledgments" environment
%% (and NOT an unnumbered section). This ensures the proper
%% identification of the section in the article metadata, and the
%% consistent spelling of the heading.
\begin{acknowledgments}
This research was funded by Flanders Make organization under the project OperatorAssist\_SBO, project number 2021-0133. Flanders Make is the Flemish strategic research center for the manufacturing industry in Belgium.
\end{acknowledgments}

%%
%% Define the bibliography file to be used
\bibliography{references}

\end{document}

%%
%% End of file
